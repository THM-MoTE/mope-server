\section{REST-API}

\newcommand{\returnline}[1]{\textbf{Returns:} #1}
\newcommand{\return}[2]{\textbf{Returns:} #1
  \begin{addmargin}{0.05\textwidth}
  #2
  \end{addmargin}
}
\newcommand{\param}[2]{\textbf{Request-Body:} #1
  \begin{addmargin}{0.05\textwidth}
  #2
  \end{addmargin}
}

All Requests and Responses are using \textbf{JSON-Strings} as data.

\begin{itemize}
  \item \code{POST /connect} \\
  Connects the client with the server. \\
  Awaits the content of the \code{project.json} -file as Request-Body. This file should be
  available in the root-directory of the project. \\
  \return{Unique ID for the project.}{This ID is later used to identify the project.}

  \item \code{POST /disconnect?project-id=<ID>} \\
  Disconnects the client from the server and kills all remaining processes associated
  with the project. \\
  \returnline{Status-Code: 200; empty body}

  \item \code{POST /stop-server} \\
  Kills all remaining sessions/projects and stops the server-process. \\
  \returnline{Status-Code: 200; empty body}

  \item \code{POST /compile?project-id=<ID>} \\
  Compiles the project identified by ID. \\
  \return{JSON-Array containing compiler-errors.}{Example:}
  \begin{lstlisting}[basicstyle=\small,language=json]
    //list of errors, empty if no error
    [
      <ERROR-OBJECT>,
      <ERROR-OBJECT>,
      ...
    ]
  \end{lstlisting}

  \item \code{POST /complete?project-id=<ID>} \\
  Returns code-completion suggestions. \\
  \param{Substring which should get completed and the containing file.}{Example:}
  \begin{lstlisting}[basicstyle=\small,
    language=json]
    {
      //file for completion
      file: <RelativePath>
      //line in which the string occured
      line: <Number>
      //column in which the string occured
      column: <Number>
      //substring which should get completed
      string: <String>
    }
  \end{lstlisting}
  \return{Json-Array with possible code-completions}{Example:}
  \begin{lstlisting}[basicstyle=\small,
    language=json]
    [
      <Completion-Object>,
      <Completion-Object>,
      ...
    ]
  \end{lstlisting}
\end{itemize}

Used datatypes and objects.
\begin{itemize}
  \item \textbf{Type:} \code{RelativePath} \\
  String with relative path (relative to the root-directory of the project)

  \item \textbf{Object:} \code{Completion-Object} \\
  Example:
  \begin{lstlisting}[basicstyle=\small,
    language=json]
    {
      //type of the completion
      //1 of the given strings
      completionType: "type" | "variable" | "method"
      //completions-name
      name: <String>
      //OPTIONAL: a list of parameter-names if completionType = "method"
      parameters: [
        <String>,
        <String>,
        ...
      ]
    }
  \end{lstlisting}

  \item \textbf{Object:} \code{ERROR-OBJECTS} \\
  Example:
  \begin{lstlisting}[basicstyle=\small,
    language=json]
    {
      file: <RelativePath> //file, in which the error occured
      line: <Number> //Lineno of error
      column: <Number> //Columnno of error
      message: <String> //error message
    }
  \end{lstlisting}
\end{itemize}
