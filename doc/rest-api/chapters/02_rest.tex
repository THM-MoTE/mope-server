\section{REST-API}

\newcommand{\returnline}[1]{\textbf{Returns:} #1}
\newcommand{\return}[2]{\textbf{Returns:} #1
  \begin{addmargin}{0.05\textwidth}
  #2
  \end{addmargin}
}
\newcommand{\param}[2]{\textbf{Request-Body:} #1
  \begin{addmargin}{0.05\textwidth}
  #2
  \end{addmargin}
}

All Requests and Responses are using \textbf{JSON-Strings} as data.

\subsection{Connecting to the server}
Connects a client to the server and starts a manager for the given project.


% template taken from: https://gist.github.com/iros/3426278

\begin{itemize}
\item \textbf{URL} \code{/moie/connect}
\item \textbf{Method} \code{POST}
\item \textbf{URL Params} Empty
\item \textbf{Data Params}
  The content of the \code{moie-project.json} file. This file is located at the /-directory of the project.
  This object defines several project-specific configurations and looks like this:
  \begin{lstlisting}[basicstyle=\small,language=json]
  {
    "path": <String>, //absolute path to the project
    "outputDirectory": <String>, //directory for compiler-generated files
    "compilerFlags": [  //list of flags for the compiler
      <String>,
      <String>
    ],
    //OPTIONAL: relative path to the mos-script that's used as default-script
    //if empty ``build.mos'' will be assumed
    "buildScript": <String>
  }
  \end{lstlisting}
  \decodedAs{de.thm.moie.project.ProjectDescription}
\item \textbf{Success Response}
  \newline\textbf{Code} 200 OK
  \newline\textbf{Content} ID which identifies the project. For example: \code{1}
\item \textbf{Error Response}
  \newline\textbf{Cause} The data-field doesn't contain a valid object.
  \newline\textbf{Code} 400 Bad Request
  \newline\textbf{Content} Akka-generated error message which states that it couldn't decode the
  object. Similar like this: \code{The request content was malformed..}

  \textbf{Cause} The data-field isn't a valid ProjectDescription.
  The fields of the given object have to meet this criterias:
  \begin{itemize}
  \item The \code{path}-field must be a fullpath and a directory
  \item The optional \code{buildScript}-field must be a regular file with a ``.mos'' extension.
  \end{itemize}
  \textbf{Code} 400 Bad Request
  \newline\textbf{Content} An json-array containing validation errors.
  \item \textbf{Sample Call}
\end{itemize}

\subsection{Compiling a project}
Compiles all files from the project identified by the given \code{id}.

\begin{itemize}
\item \textbf{URL} \code{/moie/project/:id/compile}
\item \textbf{Method} \code{POST}

\item \textbf{URL Params}
  \newline\textbf{Required} \code{id=[integer]}

\item \textbf{Data Params} An \code{FilePath}-object with a path to the currently opened file.
  

\item \textbf{Success Response}
  \newline\textbf{Code} 200 OK
  \newline\textbf{Content} An array containing compiler-errors.
  \begin{lstlisting}[basicstyle=\small,language=json]
  [
    <Error-Object>,
    <Error-Object>,
    ...
  ]
  \end{lstlisting}
  
\item \textbf{Error Response}
  \newline\textbf{Cause} Given id is unknown.
  \newline\textbf{Code} 404 Not Found
  \newline\textbf{Content} \code{unknown project-id :id}

  \fixedspace\textbf{Cause} Given filepath is unknown.
  \newline\textbf{Code} 404 Not Found
  \newline\textbf{Content} \code{Can't find file :path}

\item \textbf{Sample Call}
\end{itemize}

\subsection{Compiling a script}
Compiles a Modelica Script from the project identified by the given \code{id}.

\begin{itemize}
\item \textbf{URL} \code{/mope/project/:id/compileScript}
\item \textbf{Method} \code{POST}

\item \textbf{URL Params}
  \newline\textbf{Required} \code{id=[integer]}

\item \textbf{Data Params} An \code{FilePath}-object with a path to the script which
  should get compiled.

\item \textbf{Success Response}
  \newline\textbf{Code} 200 OK
  \newline\textbf{Content} An array containing compiler-errors.
  \begin{lstlisting}[basicstyle=\small,language=json]
  [
    <Error-Object>,
    <Error-Object>,
    ...
  ]
  \end{lstlisting}

\item \textbf{Error Response}
  \newline\textbf{Cause} Given id is unknown.
  \newline\textbf{Code} 404 Not Found
  \newline\textbf{Content} \code{unknown project-id :id}

  \fixedspace\textbf{Cause} Given filepath is unknown.
  \newline\textbf{Code} 404 Not Found
  \newline\textbf{Content} \code{Can't find file :path}
\item \textbf{Sample Call}
\end{itemize}

\subsection{Disconnecting from the server}
Disconnects a client identified by \code{id} from the server. This stops the
manager of the project if there is no user for the project left.
If this is the last connected client and the flag \code{exitOnLastDisconnect}
in the server configuration file is active the whole server-process stops.

\begin{itemize}
\item \textbf{URL} \code{/moie/project/:id/disconnect}
\item \textbf{Method} \code{POST}

\item \textbf{URL Params}
  \newline\textbf{Required} \code{id=[integer]}

\item \textbf{Data Params} Empty

\item \textbf{Success Response}
  \newline\textbf{Code} 204 No Content
  \newline\textbf{Content} Empty

\item \textbf{Error Response}
  \newline\textbf{Cause} Given id is unknown.
  \newline\textbf{Code} 404 Not Found
  \newline\textbf{Content} \code{unknown project-id :id}

\item \textbf{Sample Call}
\end{itemize}

\subsection{Stopping the server}
Stops the server-process discarding all running projects.

\begin{itemize}
\item \textbf{URL} \code{/moie/stop-server}
\item \textbf{Method} \code{POST}

\item \textbf{URL Params} Empty

\item \textbf{Data Params} Empty

\item \textbf{Success Response}
  \newline\textbf{Code} 202 Accepted
  \newline\textbf{Content} Empty

\item \textbf{Error Response} Empty
%  \newline\textbf{Cause} Description
%  \newline\textbf{Code} 400 Bad Request
%  \newline\textbf{Content} MESSAGE

\item \textbf{Sample Call}
\end{itemize}

\subsection{Common type definitions}
The following types are used to describe the request and response data
from the server.

\subsubsection{Error-Object}
Describes compiler-errors in Modelica files or Modelica Script-files.
\newline\decodedAs{de.thm.moie.compiler.CompilerError}
  \begin{lstlisting}[basicstyle=\small,language=json]
    {
      "type": "Error" | "Warning", //type of the message from the compiler
      "file": <String>, //absolute path to the file which contains the error
      "start": <File-Position>, //Start-point of error
      "end": <File-Position>, //End-point of error
      "message": <String> //Error-message from the compiler
    }
  \end{lstlisting}

\subsubsection{FilePath}
A path to a file.
\newline\decodedAs{de.thm.moie.position.FilePath}
\begin{lstlisting}[basicstyle=\small,language=json]
  {
    "path": <String> //absolute path to the script
  }
\end{lstlisting}

\subsubsection{File-Position}
Describes a position inside of a file. Contains a
line and a column.
\newline\decodedAs{de.thm.moie.position.FilePosition}
  \begin{lstlisting}[basicstyle=\small,language=json]
  {
    "line": <number>, //line number
    "column": <number> //column number
  }
  \end{lstlisting}

  \subsubsection{FileWithLine}
  A path to a file and the line containing the searched value.
  \newline\decodedAs{de.thm.moie.position.FileWithLine}
  \begin{lstlisting}[basicstyle=\small,language=json]
    {
      "path": <String>, //absolute path to the file
      "line": <number> //line number
    }
  \end{lstlisting}

  \subsubsection{Completion-Object}
  Describes possible code completions.
  \newline\representedBy{de.thm.moie.suggestion.CompletionResponse}
  \begin{lstlisting}[basicstyle=\small,language=json]
    {
      //type of completion; 1 of the listed strings
      "completionType": "Type" | "Variable" | "Function" | "Keyword" | "Package" | "Model" | "Class",
      "name": <String>, //the completion for the given word
      //OPTIONAL: list containing names of parameters if completionType=function
      "parameters": [
        <String>,
        <String>,
        ...
      ],
      //OPTIONAL: the class comment describing the name attribute
      "classComment": <String>,
      //OPTIONAL: the `Modelica-type` of the name
      "type": <String>
    }
  \end{lstlisting}

  \subsubsection{ClassComment}
  A comment of a class or symbol.
  \newline\representedBy{de.thm.moie.doc.ClassComment}
  \begin{lstlisting}[basicstyle=\small,language=json]
    {
      "className": <String> //the name of the class/symbol
      "comment": <String> //the comment string
    }
  \end{lstlisting}

  \subsubsection{TypeOf}
  The type of a property.
  \newline\representedBy{de.thm.moie.suggestion.TypeOf}
  \begin{lstlisting}[basicstyle=\small,language=json]
    {
      "name": <String>, //name of property
      "type": <String>, //type of property
      //OPTIONAL: property's comment
      "comment": <String>
    }
  \end{lstlisting}

\newpage
\newpage
\begin{itemize}
  \item \code{POST /connect} \\
  Connects the client with the server. \\
  Awaits the content of the \code{project.json} -file as Request-Body. This file should be
  available in the root-directory of the project. \\
  \return{Unique ID for the project.}{This ID is later used to identify the project.}

  \item \code{POST /disconnect?project-id=<ID>} \\
  Disconnects the client from the server and kills all remaining processes associated
  with the project. \\
  \returnline{Status-Code: 200; empty body}

  \item \code{POST /stop-server} \\
  Kills all remaining sessions/projects and stops the server-process. \\
  \returnline{Status-Code: 200; empty body}

  \item \code{POST /compile?project-id=<ID>} \\
  Compiles the project identified by ID. \\
  \return{JSON-Array containing compiler-errors.}{Example:}
  \begin{lstlisting}[basicstyle=\small,language=json]
    //list of errors, empty if no error
    [
      <Error-Object>,
      <Error-Object>
      ...
    ]
  \end{lstlisting}

  \item \code{POST /complete?project-id=<ID>} \\
  Returns code-completion suggestions. \\
  \param{Substring which should get completed and the containing file.}{Example:}
  \begin{lstlisting}[basicstyle=\small,
    language=json]
    {
      //file for completion
      file: <RelativePath>
      //line in which the string occured
      line: <Number>
      //column in which the string occured
      column: <Number>
      //substring which should get completed
      string: <String>
    }
  \end{lstlisting}
  \return{Json-Array with possible code-completions}{Example:}
  \begin{lstlisting}[basicstyle=\small,
    language=json]
    [
      <Completion-Object>,
      <Completion-Object>,
      ...
    ]
  \end{lstlisting}
\end{itemize}

Used datatypes and objects.
\begin{itemize}
  \item \textbf{Type:} \code{RelativePath} \\
  String with relative path (relative to the root-directory of the project)

  \item \textbf{Object:} \code{Completion-Object} \\
  Example:
  \begin{lstlisting}[basicstyle=\small,
    language=json]
    {
      //type of the completion
      //1 of the given strings
      completionType: "type" | "variable" | "method" | "keyword"
      //completions-name
      name: <String>
      //OPTIONAL: a list of parameter-names if completionType = "method"
      parameters: [
        <String>,
        <String>,
        ...
      ]
    }
  \end{lstlisting}

  \item \textbf{Object:} \code{Error-Object} \\
  Example:
  \begin{lstlisting}[basicstyle=\small,
    language=json]
    {
      file: <RelativePath> //file, in which the error occured
      line: <Number> //Lineno of error
      column: <Number> //Columnno of error
      message: <String> //error message
    }
  \end{lstlisting}
\end{itemize}
