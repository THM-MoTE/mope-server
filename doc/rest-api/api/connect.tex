\subsection{Connecting to the server}
Connects a client to the server and starts a manager for the given project.


% template taken from: https://gist.github.com/iros/3426278

\begin{itemize}
\item \textbf{URL} \code{/moie/connect}
\item \textbf{Method} \code{POST}
\item \textbf{URL Params} Empty
\item \textbf{Data Params}
  The content of the \code{moie-project.json} file. This file is located at the /-directory of the project.
  This object defines several project-specific configurations and looks like this:
  \begin{lstlisting}[basicstyle=\small,language=json]
  {
    "path": <String>, //absolute path to the project
    "outputDirectory": <String>, //directory for compiler-generated files
    "compilerFlags": [  //list of flags for the compiler
      <String>,
      <String>
    ],
    //OPTIONAL: relative path to the mos-script that's used as default-script
    //if empty ``build.mos'' will be assumed
    "buildScript": <String>
  }
  \end{lstlisting}
  \decodedAs{de.thm.moie.project.ProjectDescription}
\item \textbf{Success Response}
  \newline\textbf{Code} 200 OK
  \newline\textbf{Content} ID which identifies the project. For example: \code{1}
\item \textbf{Error Response}
  \newline\textbf{Cause} The data-field doesn't contain a valid object.
  \newline\textbf{Code} 400 Bad Request
  \newline\textbf{Content} Akka-generated error message which states that it couldn't decode the
  object. Similar like this: \code{The request content was malformed..}
  \item \textbf{Sample Call}
\end{itemize}
